\let\safetikz\shipout
\input pgf
\input tikz
\usetikzlibrary{calc,matrix}
\tikzset{>=stealth}
\let\shipout\safetikz
\input stmary

\magnification\magstephalf
\parskip3pt
\baselineskip14pt

\catcode`@=11
\def\oldstyle{\fam\@ne\teni}

\def\bbbb{{\rm I\!B}}

\def\slug{\hbox{\kern1.5pt\vrule width2.5pt height6pt depth1.5pt\kern1.5pt}}
\def\slugonright{\vrule width0pt\nobreak\hfill\slug}

% Equations
\def\eqsimgraph{1}
\def\eqex{2}
\def\eqexadjmat{3}
\def\eqmap{4}
\def\eqexva{5}
\def\eqfuncsem{6}
\def\eqonehota{7}
\def\eqonehotb{8}
\def\eqinjsem{9}
\def\eqonehotc{10}
\def\eqedgep{11}
\def\eqedge{12}

\centerline{\bf Reverse engineering using simulation graphs and bipartite subgraph isomorphism}
\centerline{--- {\it Preliminary notes on our discussions\/} ---}
\centerline{Mathias Soeken}
\centerline{September 4 -- \dots, 2014}
\bigskip\bigskip

\noindent{\bf 1. Simulation graphs.}  Let $y=f(x)$ be a multibit function which
is defined over $m$ functions $f_1(x_1,\dots,x_n), \dots, f_m(x_1,\dots,x_n)$.
In other words, $y=f_1\dots f_m$ is a binary vector of length $m$ and
$x=x_1\dots x_n$ is a binary vector of length $n$.  Given such a multibit
function together with simulation vectors $x^{(1)},\dots, x^{(k)}$ a {\it
simulation graph\/} is a directed bigraph $G=(V,A)$ with source vertices
$v_1,\dots,v_k$ for each simulation vector and sink vertices $v_{k+1}, \dots,
v_{k+m}$ for each output of $f$.  It has an arc
$$ v_i \longrightarrow v_{k+j}, \quad \hbox{if and only if} \quad f_j(x^{(i)})=1\eqno(\eqsimgraph) $$
for $1\le i\le k$ and $1\le j\le m$.  If $f$ describes a 2-bit adder with 4
input variables and 3 output variables, the simulation graph for all patterns
that consist of a single ones, two ones, and all ones looks as follows:
$$
\font\sevenrm=cmr7
\tikzpicture
  \foreach \y/\p in {0/0001,1/0010,2/0100,3/1000,4/0011,5/0101,6/0110,7/1001,8/1010,9/1100,10/1111} {%
    \coordinate (sim\p) at ($(0,0)-.35*(0,\y)$);
    \fill (sim\p) circle (2pt);
    \node[left,inner sep=1pt,font=\sevenrm] at ([xshift=-4pt] sim\p) {\p};
  }
  \coordinate (y1) at ([xshift=4cm] $(sim0001)!.3!(sim1111)$);
  \fill (y1) circle (2pt);
  \node[right,inner sep=1pt,font=\sevenrm] at ([xshift=4pt] y1) {$y_1$};
  \coordinate (y2) at ([xshift=4cm] $(sim0001)!.5!(sim1111)$);
  \fill (y2) circle (2pt);
  \node[right,inner sep=1pt,font=\sevenrm] at ([xshift=4pt] y2) {$y_2$};
  \coordinate (y3) at ([xshift=4cm] $(sim0001)!.7!(sim1111)$);
  \fill (y3) circle (2pt);
  \node[right,inner sep=1pt,font=\sevenrm] at ([xshift=4pt] y3) {$y_3$};

  \draw (sim0001) to[bend left=30] (y1);
  \draw (sim0010) to[bend left=25] (y2);
  \draw (sim0100) to[bend left=5] (y1);
  \draw (sim1000) to[bend left=5] (y2);
  \draw (sim0011) to (y1);
  \draw (sim0011) to (y2);
  \draw (sim0101) to (y2);
  \draw (sim0110) to (y1);
  \draw (sim0110) to (y2);
  \draw (sim1001) to (y1);
  \draw (sim1001) to (y2);
  \draw (sim1010) to (y3);
  \draw (sim1100) to[bend right=5] (y1);
  \draw (sim1100) to[bend right=10] (y2);
  \draw (sim1111) to[bend right=25] (y2);
  \draw (sim1111) to[bend right=30] (y3);
\endtikzpicture
$$

\medskip\noindent {\bf 2. Bipartite subgraph isomorphism.} A {\it unidirected\/}
bigraph is a digraph $G=(V,A)$ for which $V$ is a partition of two disjoint sets
of sources $V'$ and sinks $V''$ and arcs $A\subseteq V'\times V''$, i.e.~the
initial and final vertex of each arc lies in $V'$ and $V''$, respectively.  The
adjacency matrix of such a unidirected bigraph always has the form
$M=\big({O\atop O}{M'\atop O}\big)$ where each $O$ is a matrix containing only
zeros and the rows and columns of $M$ are arranged in a way that first vertices
are taken from $V'$ and then from $V''$.  Consequently, we can drop the $O$
matrices and end up with the {\it compact\/} adjacency matrix~$M'$.  Given two
unidirected bigraphs called {\it target\/} $G=(V,A)$ and {\it pattern\/}
$P=(W,B)$, the {\it bipartite subgraph isomorphism\/} problem asks whether there
exists a subgraph $G'$ of $G$ such that~$G'$ is isomorphic to $P$, denoted
$G'\cong P$.  In the remainder, we will use the graphs
$$
  G=\quad
  \tikzpicture[baseline=(c.base),every node/.style={circle,draw,minimum size=.5cm,inner sep=1pt}]
    \node (v1) at (0,0) {$1$};
    \node (v2) at (0,-.75) {$2$};
    \node (v3) at (0,-1.5) {$3$};
    \node (v4) at (0,-2.25) {$4$};
    \node (w1) at ([xshift=1.5cm] $(v1)!.5!(v2)$) {$\bar 1$};
    \node (w2) at ([xshift=1.5cm] $(v2)!.5!(v3)$) {$\bar 2$};
    \node (w3) at ([xshift=1.5cm] $(v3)!.5!(v4)$) {$\bar 3$};

    \draw[->] (v1) to[bend left=10] (w1);
    \draw[->] (v1) -- (w2);
    \draw[->] (v2) -- (w1);
    \draw[->] (v2) -- (w3);
    \draw[->] (v3) -- (w2);
    \draw[->] (v4) to[bend left=10] (w1);
    \draw[->] (v4) to[bend right=10] (w2);
    \coordinate (c) at (current bounding box.west);
  \endtikzpicture
  \qquad\hbox{and}\qquad
  P=\quad
  \tikzpicture[baseline=(c.base),every node/.style={circle,draw,minimum size=.5cm,inner sep=1pt}]
    \node (v1) at (0,0) {$1$};
    \node (v2) at (0,-.75) {$2$};
    \node (v3) at (0,-1.5) {$3$};
    \node (w1) at ([xshift=1.5cm] $(v1)!.5!(v2)$) {$\bar 1$};
    \node (w2) at ([xshift=1.5cm] $(v2)!.5!(v3)$) {$\bar 2$};

    \draw[->] (v1) to[bend left=10] (w1);
    \draw[->] (v1) to[bend right=5] (w2);
    \draw[->] (v2) -- (w1);
    \draw[->] (v3) -- (w1);
    \draw[->] (v3) to[bend right=10] (w2);

    \coordinate (c) at (current bounding box.west);
  \endtikzpicture
  \eqno(\eqex)
$$
as target and pattern in a running example.  Their compact adjacency matrices
are
$$
  G_{\rm A} = \pmatrix{1 & 1 & 0 \cr 1 & 0 & 1 \cr 0 & 1 & 0 \cr 1 & 1 & 0}
  \qquad\hbox{and}\qquad
  P_{\rm A} = \pmatrix{1 & 1 \cr 1 & 0 \cr 1 & 1}
  \eqno(\eqexadjmat)
$$
with entries $g_{i,k}$ and $p_{j,l}$, respectively.  It can readily be verified,
that there exists a subgraph isomorphism between these two graphs.

\noindent {\bf SAT formulation.} We will now create a Boolean function $f$ that
is satisfiable if and only if $P$ is isomorphic to a subgraph of $G$.  In case
there exists such a subgraph it can be obtained from a satisfying assignment to
$f$.  We assume that $G$ has $n+q$ vertices; $n$ sources and $q$ sinks.
Analogously, $P$ has $m+r$ vertices.  Obviously, we have $m\le n$ and $r\le q$.

First, we are looking to the bipartite subgraph isomorphism problem from a
different angle.  Let $(g_{i,k})$ and $(p_{j,l})$ be the compact adjacency
matrices of the target $G$ and pattern $P$, respectively.  The existence of a
bipartite subgraph isomorphism is equivalent to the existence of two injective
functions $v:[m]\to[n]$ and $a:[r]\to[q]$ such that
$$ p_{j,l}=g_{v(j),a(l)} \eqno(\eqmap) $$
for all $1\le j\le m$ and $1\le l\le r$.  (The notation $[n]$ is a shorthand for
the set $\{1,\dots,n\}$.)  The function $v$ maps each source in $P$ to a
distinct source in $G$, while the function $a$ maps each sink in $P$ to a
distinct sink in $G$.  For the example in (\eqex) we have
$$
  v(1)=4,\qquad v(2)=3,\qquad v(3)=1,\qquad
  a(\bar 1)=\bar 2,\qquad a(\bar 2)=\bar 1.
  \eqno(\eqexva)
$$
(The barred notation for values in $a$ has intentionally been used to improve
comprehensibility.)

This equivalent formulation makes it easy for us to describe the satisfiability
function $f$.  The function $v$ is described by $mn$ variables $v_{j,i}$ such
that $v(j)=i$ if and only if $v_{j,i}=1$ for all $1\le j\le m$ and $1\le i\le
n$; in other words we have $v_{j,v(j)}=1$.  To ensure function semantics,
i.e.~each $v(j)$ is assigned exactly one value, clauses to express
$$ v_{j,1}+v_{j,2}+\cdots+v_{j,n}=1 \qquad\hbox{for $1\le j\le m$} \eqno(\eqfuncsem) $$
are added.  The encoding is sometimes referred to as {\it one hot encoding\/}
and can be enforced by the clauses
$$ v_{j,1}\lor\cdots\lor v_{j,n} \qquad \hbox{for $1\le j\le m$} \eqno(\eqonehota) $$
to ensure that at least one bit is set to 1 together with the exclusion clauses
$$ \bar v_{j,i}\lor \bar v_{j,i'} \qquad \hbox{for $1\le j\le m$ and $1\le i<i'\le n$} \eqno(\eqonehotb) $$
that ensure that at most one bit is set to 1.
The injectivity property can be enforced by adding clauses expressing
$$ v_{1,i}+v_{2,i}+\cdots+v_{m,i} \le 1 \qquad\hbox{for $1\le i\le n$}. \eqno(\eqinjsem) $$
This ensures that for each $i\in\{1,\dots,n\}$ there is at most one $j$ such
that $v(j)=i$ and can be encoded using only the exclusion clauses
$$ \bar v_{j,i}\lor \bar v_{j',i} \qquad \hbox{for $1\le j<j'\le m$ and $1\le i\le n$}. \eqno(\eqonehotc) $$
Analogously, $rq$ variables $a_{l,k}$ describe function $a$ for $1\le l\le r$
and $1\le k\le q$ and similar clauses to (\eqonehota), (\eqonehotb), and
(\eqonehotc) ensure the correct semantics.

We are now describing (\eqmap) in terms of clauses.  It can easily be seen that
the functions $v$ and $a$ must ensure the property
$$ (v(j)=i) \land (a(l)=k) \quad\Longrightarrow\quad g_{i,k}=p_{j,l}. \eqno(\eqedgep) $$
In other words, when $v$ maps vertex $j$ in $P$ to vertex $i$ in $G$ and when
$a$ maps vertex $l$ in $P$ to vertex $k$ in $G$, then we either have $i
\longrightarrow k$ and $j \longrightarrow l$ or we have $i
\longarrownot\longrightarrow k$ and $j \longarrownot\longrightarrow l$.  By
changing the direction of (\eqedgep) and using the variables that encode $v$ and
$a$ we see that for all $i,j,k$ and $l$ we add the clause
$$ (\bar v_{j,i} \lor \bar a_{l,k}) \qquad \hbox{if $g_{i,k}\neq p_{j,l}$}. \eqno(\eqedge) $$
A similar encoding has been proposed for general subgraph isomorphism by
C\v{a}lin Anton and Lane Olson in [{\sl Canadian AI\/ \bf 22} (2009), 16--26].
Our formalism allows a smaller number of variables since vertices are
partitioned into two disjoint sets which corresponds to bounded color
multiplicity subgraph isomorphism (see e.g.~[V.~Arvind, P.~P.~Kurur, and
T.~C.~Vijayaraghavan, {\sl CoCo\/ \bf 20} (2005), 13--27]).  With minor
adjustments the encoding can also be used for graph isomorphism as shown by
Jacobo Tor\'an in [{\sl SAT\/ \bf 16} (2013), 52--66].

\medskip\noindent{\bf \llap{*}Complexity of the formula.}  Now we want to look
into the size of the resulting formulas.  There are in total $mn+rq+nmqr\le
O(|V|^4)$ variables $v_{j,i}$, $a_{l,k}$, and $h_{i,j,k,l}$.  The total number
of clauses is
$$
  \kappa(G,P)=
  \underbrace{m+r}_{(\eqonehota)}
  + \underbrace{\left(m\atop 2\right)+\left(r\atop 2\right)}_{(\eqonehotb)}
  + \underbrace{\left(n\atop 2\right)+\left(q\atop 2\right)}_{(\eqonehotc)}
  + \underbrace{(nq-|A|)\cdot|B|+|A|\cdot(mr-|B|)}_{(\eqedge)},
$$
where for (\eqonehota), (\eqonehotb), and (\eqonehotc) also the clauses to
encode function $a$ are taken into account.  Only the last term is not obvious.
Matrix $G_{\rm A}$ has $|A|$ ones and $nq-|A|$ zeros and matrix $P_{\rm A}$ has
$|B|$ ones and $mr-|B|$ zeros.  Now each element of matrix $G_{\rm A}$ is
compared to each element of matrix $P_{\rm A}$ and two cases can occur. In case
the element is $0$, which is the case for $(nq-|A|)$ entries, one needs to add a
clause for each element that is $1$ in matrix $P_{\rm A}$ which are $|B|$ many.
A similar argument hods for the case that the element is $1$.

Considering that $|B|\le |A|$, $m\le n$, and $r\le q$, we have
$$
  \eqalign{
  \kappa(G,P) &\le
  n+q
  + 2\cdot\left(n\atop 2\right)+2\cdot\left(q\atop 2\right)
  + 2\cdot (nq-|A|)\cdot |A|.}
$$

\medskip\noindent{\bf Preprocessing.}


\medskip\noindent{\bf Further reading.}  The subgraph isomorphism problem is
known to be NP-complete for general graphs [M.\ R.\ Garey and D.\ S.\ Johnson,
{\sl Computers and Intractibility} (1979)].  Effecient implementations for
special cases are discussed in [J.\ E.\ Hopcroft and J.\ K.\ Wong, {\sl STOC\/
\bf 6} (1974), 172--184] and [E.\ M.\ Luks, {\sl JCSS\/ \bf 25} (1982), 42--65].

[C.\ Solnon, {\sl AI\/ \bf 174} (2010), 850--864]  [S.\ Zampelli and C.\ Solnon, {\sl Constraints\/ \bf 15} (2010), 327--353]

\centerline{****}

\noindent{\bf Consideration of inverters.}

\centerline{****}

\noindent{\bf Symmetry breaking.}

\centerline{****}

\noindent{\bf Further improvements with preprocessing.}

\centerline{****}

\noindent{\bf Experiments.}

\topinsert
\noindent\hfil
\vbox{
\offinterlineskip
\hrule
\tabskip=0em plus2em minus.5em
\halign{\strut\enspace#\hfil & \hfil# & \hfil# & \hfil# & \hfil# & \hfil# \cr
  Benchmark & MiniSAT & SAT13 & LAD & MiniSAT (inv.) & SAT13 (inv.) \cr
  \noalign{\hrule}
  b1-4 &   0.12 &  0.00 & 0.00 & 111.32 &   67.58 \cr
  b1-5 &   0.27 &  0.01 & 0.01 &        & 1161.11 \cr
  b1-6 &  22.20 &  0.21 & 0.02 \cr
  b1-7 &   2.14 &  0.14 & 0.04 \cr
  b1-8 & 614.48 & 11.77 & 0.08 \cr
  b2-4 &   0.13 &  0.09 & 0.00 \cr
  b2-5 &   0.85 &  0.41 & 0.00 \cr
  b2-6 &   2.19 &  1.86 & 0.00 \cr
  b2-7 &   5.93 &  5.88 & 0.00 \cr
  b2-8 &  13.41 & 16.24 & 0.00 \cr
}
\hrule}
\endinsert

\centerline{****}

\noindent{\bf Further reading.}
[J.\ A.\ Roy, I.\ L.\ Markov, and V.\ Bertacco, {\sl IWLS\/ \bf 13} (2004)]

\centerline{****}

\noindent{\bf AND-inverter chains.} For a function of $n$ variables
$(x_1,\dots,x_n)$ let $(x_{n+1},\dots,x_{n+r})$ be a Boolean chain as defined in
[TAOCP, Sect.~7.1.2].  Index $i\in\{n+1,\dots,n+r\}$ is called an {\it inner
operation\/} if there exists an index $i'>i$ such that $j(i')=i$ or if there
exists an index $i''>i$ such that $k(i'')=i$.  In other words, the outcome of
operation $i$ is used as input by some later operation.  An index that is not an
inner operation is called {\it outer operation}.  For Boolean chains that
represent a Boolean function, index $n+r$ is the only outer operation.  An {\it
AND-inverter chain\/} is a Boolean chain $(x_{n+1},\dots,x_{n+r})$ with the
property that $\circ_i\in\{\land,>,<,\bar\lor\}$ for all inner operations $i$
and $\circ_i\in\{\lor,\ge,\le,\bar\land\}$ for all outer operations $i$.  Also
we may have $\circ_{n+1}=\bot$ with $j(n+1)$ and $k(n+1)$ being undefined to
represent the constant $0$ input.


\bye
