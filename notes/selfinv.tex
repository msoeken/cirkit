\magnification\magstephalf
\parskip3pt
\baselineskip14pt

\centerline{\bf Self-inverse functions and palindromic circuits}
\centerline{Mathias Soeken}
\centerline{September 24, 2014}

\bigskip
\bigskip

\dots TODO: definition of palindromic circuit --- definition of self-inverse function \dots

\bigskip

\noindent{\bf Lemma 1.} \sl Every palindromic circuit represents a self-inverse
function.\rm

\smallskip \noindent{\bf Conjecture 1.} \sl Every self-inverse function can be
realized by a palindromic circuit.\rm

\smallskip \noindent{\bf Lemma 2.} \sl A palindromic circuit has an even number
of gates, if and only if it realizes the identity function. \rm

\smallskip \noindent{\bf Open problem.} \sl When synthesizing a self-inverse
function with the Young subgroup algorithm, is there always an assignment of the
variable ordering and the 0/1 assignments, such that a palindromic circuit
results. \rm

\bye
