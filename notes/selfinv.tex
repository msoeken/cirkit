\magnification\magstephalf
\parskip3pt
\baselineskip14pt

\def\slug{\hbox{\kern1.5pt\vrule width2.5pt height6pt depth1.5pt\kern1.5pt}}
\def\slugonright{\vrule width0pt\nobreak\hfill\slug}

\centerline{\bf Self-inverse functions and palindromic circuits}
\centerline{--- {\it Preliminary notes\/} ---}
\centerline{Mathias Soeken}
\centerline{September 24 -- \dots, 2014}

\bigskip
\bigskip

\noindent{\bf 1. Introduction.}\enspace For a {\it self-inverse\/} reversible
function $f$ we have $f(f(x))=x$ for all input assignments $x$.  In the
remainder, we just use the name self-inverse function and say that a
self-inverse function $f$ is {\it trivial\/} if $f$ is the identity function.  A
circuit $C=g_1g_2\dots g_k$, which consists of mixed-polarity
multiple-controlled Toffoli gates $g_i$, is called {\it palindromic\/} if $ g_i
= g_{k+1-i}$ for all $i \in \{1,\dots,\lfloor{k\over 2}\rfloor\}$.  The circuit
is called {\it even\/} if $k$ is even and {\it odd\/} otherwise.  The first
simple observation that is described by Lemma~1 follows from the fact that every
gate $g_i$ is self-inverse.

\noindent{\bf Lemma 1.} \sl Every palindromic circuit represents a self-inverse
function.\rm

\smallskip\noindent However, the inverse direction of that lemma is not so
obvious and remains a conjecture so far.

\smallskip \noindent{\bf Conjecture 1.} \sl Every self-inverse function can be
realized by a palindromic circuit.\rm

\smallskip\noindent For all interesting cases of self-inverse functions,
i.e.~for all functions that are not the identity, a palindromic circuit must be
odd if it exists.

\smallskip \noindent{\bf Lemma 2.} \sl A palindromic circuit is even if and only
if it realizes the identity function. \rm

\smallskip\noindent {\it Proof.} Let $C=g_1g_2\dots g_{2k}$ be an even
palindromic circuit.  Then $g_{k}=g_{2k+1-k}=g_{k+1}$ and $C$ is functionally
equivalent to the circuit $g_1g_2\dots g_{k-1}g_{k+2} \dots g_{2k}$.  Again, the
two ``middle gates'' are equal and therefore we can continue this procedure
until no gates remain and we have that $C$ represents the identity function.

Now let $C=g_1g_2\dots g_kg_{k+1}g_{k+2}\dots g_{2k+1}$ be an odd palindromic
circuit and let $h$ be the function that is represented by the first $k$ gates
$g_1g_2\dots g_k$.  Since $C$ is palindromic the last $k$ gates $g_{k+2}\dots
g_{2k+1}$ represents the function $h^{-1}$.  Since $g_{k+1}$ is a Toffoli gate
there exists some $x$ such that $g_{k+1}(x)=y\neq x$.  Propagating $x$ to the
left yields $h^{-1}(x)$ at the inputs of $C$ and propagating $y$ to the right
yiels $h^{-1}(y)$ at the outputs of $C$.  Since $x\neq y$ we also have
$h^{-1}(x)\neq h^{-1}(y)$ and hence $C$ does not represent the identity
function.\qquad\slug

\smallskip \noindent{\bf Open problem 1.} \sl Assuming that there exists a
palindromic circuit for each self-inverse function, the question is whether
there also exists a palindromic circuit in a V-shape for each self-inverse
function. \rm

\medskip\noindent{\bf 2. Connection to permutations.}\enspace New insight can be
gained by investigating the respective permutations, i.e.~elements from the
symmetric group $S_{2^n}$, instead of self-inverse functions on $n$ variables.
In the literature a self-inverse permutation or self-conjugate permutation is
also often called an {\it involution\/}.  The permutation that represents the
identity is denoted $\pi_{\rm id}$.  The permutation matrix of an involution is
symmetric.

\smallskip \noindent{\bf Lemma 3.} \sl Let $f$ be a self-inverse function and
$\pi_f$ its corresponding permutation.  Then, the cycle representation of
$\pi_f$ consists only of transpositions or fixpoints. \rm

\smallskip\noindent {\it Proof.} The cycle representation is unique when
disregarding order of cycles and order of elements within cycles.  Assume that
the cycle representation of $\pi_f$ consists of a cycle $(i_1\; i_2\; \ldots \;
i_k)$ with $k>2$.  Then $\pi_f^{-1}$ consists of the cycle $(i_k \; \ldots \;
i_2 \; i_1)$ and hence $\pi_f\neq \pi_f^{-1}$. \qquad\slug

\medskip\noindent From the book of T.~Muir [{\sl A Treatise on the Theory of
Determinants\/} (1960)] we have that the number of involutions on $n$ elements
is
$$ I(n) = 1 + \sum_{k=0}^{\lceil(n-1)/2\rceil}{1\over{(k+1)!}}
          \prod_{i=0}{k}\left(\matrix{ n-2i \cr 2}\right), \eqno(1) $$ or
alternative according to S.~Skiena [{\sl Implementing Discrete Mathematics\/}
(1990)] we have
$$ I(n) = n!\sum_{k=0}^{\lceil n/2\rceil}{1\over{2^kk!(n-2k)!}}. \eqno(2) $$
The number of self-inverse functions on $n$ variables is $I(2^n)$.  From (2) it
can easily be seen that the percentage of self-inverse functions with respect to
all reversible functions is
$$ {I(2^n)\over{2^n!}} = \sum_{k=0}^{2^{n-1}}{1\over{2^kk!(2^n-2k)!}}. \eqno(3) $$
Working with involutions, we can find an equivalent conjecture to Conjecture~1
based on the notation of permutations.  First, we prove a further lemma.

\smallskip\noindent{\bf Lemma 4.} \sl A permutation $\pi$ is an involution if
and only if it can be written as $\pi = \pi_1\circ\pi_2\circ\pi_1^{-1}$ where
$\pi_1\neq \pi_{\rm id}$ and $\pi_2\neq \pi$ is an involution. \rm

\smallskip\noindent {\it Proof.}  Let $\pi = \pi_1\circ\pi_2\circ\pi_1^{-1}$ and
let $\pi_1(x)=y$, $\pi_2(y)=z$, and $\pi_1^{-1}(z)=w$.  Since $\pi_2$ is an
involution we have $\pi(x)=w$ and $\pi(w)=x$ and therefore $\pi$ is also an
involution.

Now assume that $\pi$ is an involution.  From the previous proof direction we
obtain another involution $\pi_2\neq \pi$ by taking any permutation $\pi_1\neq
\pi_{\rm id}$ by calculating $\pi_2=\pi_1\circ\pi\circ \pi_1^{-1}$.  Multiplying
the left side of the equation with $\pi_1^{-1}$ and the right side with $\pi_1$
we obtain $\pi=\pi_1^{-1}\circ\pi_2\circ\pi_1$, concluding the
proof. \qquad\slug

\smallskip\noindent {\bf Conjecture 2.} \sl Every involution $\pi$ can be
decomposed into $\pi = \pi_1\circ \pi_g\circ \pi_1^{-1}$ where $\pi_g$ is a
permutation that can be realized using a single Toffoli gate. \rm

\smallskip\noindent Conjecture 2 is equivalent to Conjecture 1.

\medskip\noindent {\bf 3. Synthesis of self-inverse functions.}\enspace \dots

\medskip\noindent {\bf 4. Lower and upper bounds.}\enspace \dots

\bye
