\magnification\magstephalf
\parskip3pt
\baselineskip14pt

\def\slug{\hbox{\kern1.5pt\vrule width2.5pt height6pt depth1.5pt\kern1.5pt}}
\def\slugonright{\vrule width0pt\nobreak\hfill\slug}

\centerline{\bf Self-inverse functions and palindromic circuits}
\centerline{--- {\it Preliminary notes\/} ---}
\centerline{Mathias Soeken}
\centerline{September 24 -- \dots, 2014}

\bigskip
\bigskip

\noindent{\bf 1. Introduction.}\enspace For a {\it self-inverse\/} reversible
function $f$ we have $f(f(x))=x$ for all input assignments $x$.  A circuit
$C=g_1g_2\dots g_k$ is called {\it palindromic\/} if $ g_i = g_{k+1-i}$ for all
$i \in \{1,\dots,\lfloor{k\over 2}\rfloor\}$.  The circuit is called {\it
even\/} if $k$ is even and {\it odd\/} otherwise.  The first simple observation
that is described by Lemma~1 follows from the fact that every gate $g_i$ is
self-inverse.

\noindent{\bf Lemma 1.} \sl Every palindromic circuit represents a self-inverse
function.\rm

\smallskip\noindent However, the inverse direction of that lemma is not so
obvious and remains a conjecture so far.

\smallskip \noindent{\bf Conjecture 1.} \sl Every self-inverse function can be
realized by a palindromic circuit.\rm

\smallskip\noindent For all interesting cases of self-inverse functions,
i.e.~for all functions that are not the identity, a palindromic circuit must be
odd if it exists.

\smallskip \noindent{\bf Lemma 2.} \sl A palindromic circuit is even if and only
if it realizes the identity function. \rm

\smallskip\noindent {\it Proof.} Let $C=g_1g_2\dots g_{2k}$ be an even
palindromic circuit.  Then $g_{k}=g_{2k+1-k}=g_{k+1}$ and $C$ is functionally
equivalent to the circuit $g_1g_2\dots g_{k-1}g_{k+2} \dots g_{2k}$.  Again, the
two ``middle gates'' are equal and therefore we can continue this procedure
until no gates remain and we have that $C$ represents the identity function.

Now let $C=g_1g_2\dots g_kg_{k+1}g_{k+2}\dots g_{2k+1}$ be an odd palindromic
circuit and let $h$ be the function that is represented by the first $k$ gates
$g_1g_2\dots g_k$.  Since $C$ is palindromic the last $k$ gates $g_{k+2}\dots
g_{2k+1}$ represents the function $h^{-1}$.  Since $g_{k+1}$ is a Toffoli gate
there exists some $x$ such that $g_{k+1}(x)=y\neq x$.  Propagating $x$ to the
left yields $h^{-1}(x)$ at the inputs of $C$ and propagating $y$ to the right
yiels $h^{-1}(y)$ at the outputs of $C$.  Since $x\neq y$ we also have
$h^{-1}(x)\neq h^{-1}(y)$ and hence $C$ does not represent the identity
function.\qquad\slug

\smallskip \noindent{\bf Open problem 1.} \sl Assuming that there exists a
palindromic circuit for each self-inverse function, the question is whether
there also exists a palindromic circuit in a V-shape for each self-inverse
function. \rm

\medskip\noindent{\bf 2. Connection to permutations.}\enspace New insight can be
gained by investigating the respective permutations, i.e.~elements from the
symmetric group $S_{2^n}$, instead of self-inverse functions on $n$ variables.

\smallskip \noindent{\bf Lemma 3.} \sl Let $f$ be a self-inverse function and
$\pi_f$ its corresponding permutation.  Then, the cycle representation of
$\pi_f$ consists only of transpositions or fixpoints. \rm

\smallskip\noindent {\it Proof.} The cycle representation is unique when
disregarding order of cycles and order of elements within cycles.  Assume that
the cycle representation of $\pi_f$ consists of a cycle $(i_1\; i_2\; \ldots \;
i_k)$ with $k>2$.  Then $\pi_f^{-1}$ consists of the cycle $(i_k \; \ldots \;
i_2 \; i_1)$ and hence $\pi_f\neq \pi_f^{-1}$. \qquad\slug

\medskip\noindent With Lemma~3 we can readily count all self-inverse function on
$n$ variables, in the following referred to as $s_n$:
$$ s_n = \prod \dots \eqno(1) $$

\medskip\noindent {\bf 3. Synthesis of self-inverse functions.}\enspace \dots

\bye
