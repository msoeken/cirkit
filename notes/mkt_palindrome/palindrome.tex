% This is LLNCS.DEM the demonstration file of
% the LaTeX macro package from Springer-Verlag
% for Lecture Notes in Computer Science,
% version 2.4 for LaTeX2e as of 16. April 2010
%
\documentclass{llncs}
%
\input{Qcircuit_MKTmod}
\begin{document}

\mainmatter              % start of the contributions

%
\title{Palindromic circuits}
%
%\titlerunning{Hamiltonian Mechanics}  % abbreviated title (for running head)
%                                     also used for the TOC unless
%                                     \toctitle is used
%
\author{Michael Kirkedal Thomsen, ...}
%
%\authorrunning{Ivar Ekeland et al.} % abbreviated author list (for running head)
%
%%%% list of authors for the TOC (use if author list has to be modified)
%\tocauthor{Ivar Ekeland, Roger Temam, Jeffrey Dean, David Grove,
% Craig Chambers, Kim B. Bruce, and Elisa Bertino}
%
\institute{Bremen University, Germany,\\
\email{I.Ekeland@princeton.edu},\\ WWW home page:
\texttt{http://users/\homedir iekeland/web/welcome.html}
\and
Universit\'{e} de Paris-Sud,
Laboratoire d'Analyse Num\'{e}rique, B\^{a}timent 425,\\
F-91405 Orsay Cedex, France}

\maketitle              % typeset the title of the contribution

\begin{abstract}
This document contains the first thoughts about reversible palindromic circuits. It is exemplified by be a in depth investigation of circuits of 2 lines. It also contains some thoughts about the relation between the permutation transposition notation and rewriting of reversible circuits.

This document is not very structured.

\keywords{}
\end{abstract}
%
\section{Introduction}
By a \emph{reversible palindromic circuit} we understand a reversible circuit which sequential circuit representation forms a palindrome; i.e. is identical when reading if from left and right. 

%
\section{Permutations and Transpositions}
\begin{definition}
Permutation
\end{definition}

\begin{definition}
Transposition is a permutation of 2 elements.
\end{definition}

All permutations can be decomposed into a product of transpositions. Actually, a permutation can be decomposed into an infinite number of products. However, for a specific function all products have the same parity; i.e. is build as a product of either an even or odd number of number of transpositions.

\begin{lemma}
Decompositions in product of transpositions preserved the parity.
\end{lemma}
\begin{proof}
Known result \qed
\end{proof}


$$(b\,c)(b\,c) = Id$$

\subsection{Calculus of Transpositions}
\begin{lemma}
$$(a\,b)(b\,c) = (a\,c)(a\,b) = (b\,c)(a\,c),$$
$\epsilon(a,b,c) \neq 0$
\end{lemma}
\begin{proof}
...
\end{proof}


\begin{lemma}
$$(a\,b)(b\,c)(a\,b) = (b\,c)(a\,b)(b\,c),$$
$\epsilon(a,b,c) \neq 0$
\end{lemma}
\begin{proof}
Using Id and Lemma before
\begin{eqnarray*}
(a\,b)(b\,c)(a\,b) &=& (a\,b)(b\,c)(a\,b)(b\,c)(b\,c) \\
&=& (b\,c)(a\,c)(a\,b)(b\,c)(b\,c) \\
&=& (b\,c)(a\,c)(a\,c)(a\,b)(b\,c) \\
&=& (b\,c)(a\,b)(b\,c)
\end{eqnarray*} \qed
\end{proof}


\section{Reversible Circuit Representation using Permutation}
\begin{definition}
A reversible circuit of $n$ lines can be represented as permutation of $2^n$ elements.
\end{definition}


\section{Example: Reversible circuits of 2 lines}

We will denote the basic reversible circuit as the transpositions. Notice that we use the work \emph{circuit} as there are not necessary gates. 

\begin{multicols}{3}
\begin{eqnarray*}
(01) & 
\Qcircuit @C=3mm @R=1mm {%
    & \ctrlo{1} & \qw \\
    & \targ     & \qw
} \\
(02) &
\Qcircuit @C=3mm @R=1mm {%
    & \targ      & \qw \\
    & \ctrlo{-1} & \qw
} \\
(03) &
\Qcircuit @C=3mm @R=1mm {%
    & \targ     & \ctrlo{1} & \targ     & \qw \\
    & \ctrl{-1} & \targ     & \ctrl{-1} & \qw
} \\
(12) &
\Qcircuit @C=3mm @R=1mm {%
    & \targ     & \ctrl{1} & \targ     & \qw \\
    & \ctrl{-1} & \targ    & \ctrl{-1} & \qw
} \\
(13) &
\Qcircuit @C=3mm @R=1mm {%
    & \targ     & \qw \\
    & \ctrl{-1} & \qw
} \\
(23) &
\Qcircuit @C=3mm @R=1mm {%
    & \ctrl{1} & \qw \\
    & \targ    & \qw
} 
\end{eqnarray*}
\end{multicols}

In total there are $\sum_{i=1}^{2^n-1} i = \frac{(2^n-1)\,(2^n)}{2} = (2^{n}-1)\, 2^{n-1}$ transpositions. Of these the $n*2^{n-1}$ fully $(n-1)$-controlled Toffoli gates

The single transpositions consists of the fully controlled 



Based on the above transpositions we can define the three disjoint products.

\begin{eqnarray*}
(01)(23) & 
\Qcircuit @C=3mm @R=1mm {%
    & \ctrlo{1} & \ctrl{1} & \qw \\
    & \targ     & \targ    & \qw
} = 
\Qcircuit @C=3mm @R=1mm {%
    & \qw   & \qw \\
    & \targ & \qw
} \\
(02)(13) &
\Qcircuit @C=3mm @R=1mm {%
    & \targ      & \targ     & \qw \\
    & \ctrlo{-1} & \ctrl{-1} & \qw
} =
\Qcircuit @C=3mm @R=1mm {%
    & \targ & \qw \\
    & \qw   & \qw
} \\
(03)(12) &
\Qcircuit @C=3mm @R=1mm {%
    & \targ     & \ctrlo{1} & \targ     & \targ     & \ctrl{1} & \targ     & \qw \\
    & \ctrl{-1} & \targ     & \ctrl{-1} & \ctrl{-1} & \targ    & \ctrl{-1} & \qw
} =
\Qcircuit @C=3mm @R=1mm {%
    & \targ     & \ctrlo{1} & \ctrl{1} & \targ     & \qw \\
    & \ctrl{-1} & \targ     & \targ    & \ctrl{-1} & \qw
} =
\Qcircuit @C=3mm @R=1mm {%
    & \targ     & \qw   & \targ     & \qw \\
    & \ctrl{-1} & \targ & \ctrl{-1} & \qw
} \bigg( =
\Qcircuit @C=3mm @R=1mm {%
    & \targ & \qw \\
    & \targ & \qw
} \bigg)\\
\end{eqnarray*}



\begin{theorem}
All single target gate can be described as a product of disjoint transpositions. 
\end{theorem}



\section{Example: Reversible circuits of 3 lines}

We will denote the basic reversible circuit as the transpositions. Notice that we use the work \emph{circuit} as there are not necessary gates. 

\begin{multicols}{3}
\begin{eqnarray*}
(01) & 
\Qcircuit @C=3mm @R=1mm {%
    & \ctrlo{1} & \qw \\
    & \ctrlo{1} & \qw \\
    & \targ     & \qw
} \\
(02) &
\Qcircuit @C=3mm @R=1mm {%
    & \ctrlo{1}  & \qw \\
    & \targ      & \qw \\
    & \ctrlo{-1} & \qw 
} \\
(03) & (01)(13)(01)
 \\
(04) &
\Qcircuit @C=3mm @R=1mm {%
    & \targ      & \qw \\
    & \ctrlo{-1} & \qw \\
    & \ctrlo{-1} & \qw 
} \\
(05) & (01)(15)(01)
 \\
(06) & (04)(46)(04)
 \\
(07) & (04)(46)(67)(46)(04)
\\
(12) & (01)(02)(01)
\\
(13) &
\Qcircuit @C=3mm @R=1mm {%
    & \ctrlo{1}  & \qw \\
    & \targ      & \qw \\
    & \ctrl{-1}  & \qw 
} \\
(14) & (15)(45)(15)
\\
(15) &
\Qcircuit @C=3mm @R=1mm {%
    & \targ      & \qw \\
    & \ctrlo{-1} & \qw \\
    & \ctrl{-1}  & \qw 
} \\
(16) & ?
\\
(17) & (15)(57)(15)
\\
(23) & \Qcircuit @C=3mm @R=1mm {%
    & \ctrlo{1} & \qw \\
    & \ctrl{1}  & \qw \\
    & \targ     & \qw
} \\
(24) & (02)(04)(02)
\\
(25) & ?
\\
(26) &
\Qcircuit @C=3mm @R=1mm {%
    & \targ      & \qw \\
    & \ctrl{-1}  & \qw \\
    & \ctrlo{-1} & \qw 
} \\
(27) & (26)(67)(26)
\\
(34) & (ab)(04)(ba)
\\
(35) & (13)(15)(13)
\\
(36) & (23)(26)(23)
\\
(37) &
\Qcircuit @C=3mm @R=1mm {%
    & \targ     & \qw \\
    & \ctrl{-1} & \qw \\
    & \ctrl{-1} & \qw 
} \\
(45) &
\Qcircuit @C=3mm @R=1mm {%
    & \ctrl{1}  & \qw \\
    & \ctrlo{1} & \qw \\
    & \targ     & \qw
} \\
(46) &
\Qcircuit @C=3mm @R=1mm {%
    & \ctrl{1}   & \qw \\
    & \targ      & \qw \\
    & \ctrlo{-1} & \qw 
} \\
(47) & (46)(67)(46)
\\
(56) & (45)(46)(45)
\\
(57) &
\Qcircuit @C=3mm @R=1mm {%
    & \ctrl{1}  & \qw \\
    & \targ     & \qw \\
    & \ctrl{-1} & \qw 
} \\
(67) &
\Qcircuit @C=3mm @R=1mm {%
    & \ctrl{1} & \qw \\
    & \ctrl{1} & \qw \\
    & \targ    & \qw
}
\end{eqnarray*}
\end{multicols}

\section{Reversible Palindromic Circuits}
\begin{definition}
A reversible palindromic circuit is...
\end{definition}



\section*{Acknowledgement}
EU FP7

%
% ---- Bibliography ----
%
\begin{thebibliography}{5}

\end{thebibliography}

\end{document}
